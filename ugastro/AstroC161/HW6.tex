\documentclass[12pt]{article}
\usepackage[utf8]{inputenc}
\usepackage{graphicx}
\usepackage{amsmath}
\usepackage{float}

\title{HW6}
\author{Xin Gao}
\date{\today}

\begin{document}

\maketitle

\section{Introduction}
The first minutes of the universe was marked by rapid physical changes that led the dynamics and particle distributions of the universe that we know of today. Events of the Big Bang Nucleosynthesis resulted in the production of the first multi-baryon nuclei such as Deuterium and Lithium. Using rudimentary astrophysics and statistical mechanics, we calculate the time evolution of the quantity of particles, their relation to temperature, and other quantities such as the Hubble parameter and particle collision rates.

\section{Relevant Equations and Constants}
The primary statistics used here to determine particle densities and ratios is given by the Maxwell-Boltzmann equation, which describes the number density of particles in thermal equilibrium at a temperature T:

\begin{align}n = g(\frac{mkT}{2\pi\hbar^2})e^{\frac{-mc^2}{kT}}
\end{align}where g is the number of spin states available to the particles, c is the speed of light, m is the particle mass, and k is the boltzmann constant.
\end{document}


